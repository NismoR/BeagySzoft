%----------------------------------------------------------------------------
%\chapter{\LaTeX-eszközök}\label{sect:LatexTools}
%----------------------------------------------------------------------------
\section{Általános leírás}
%----------------------------------------------------------------------------
A megvalósítani kívánt játék egy körökre osztott stratégiai játék, mely során egy harctéren különböző fajú karakterekkel harcolnak a játékosok.
A karakterek a játék előtt meghatározott felszerelést visznek magukkal, melyeket felhasználhatják a harc során. 

Körönként egy karakter egyet lép, majd egy kisorsolt felszerelést felhasznál. A játék addig tart, amíg már csak egy játékos karaktere(i) marad(nak) életben.


%----------------------------------------------------------------------------
\section{A játék menete}
%----------------------------------------------------------------------------
\begin{enumerate}
	\item Egyik játékos elindítja a Szervert, a másik (többi) kliensként csatlakozik.
	\item Szerver játékos beállítja a játék paramétereit.
	\begin{itemize}\label{StartParams}
		\item Maximális játékos szám, maximális karakter száma játékosonként.
		\item Maximális pályaméret (szélesség és magasság).
		\item Használható karakter fajok
		\begin{itemize}
			\item Azok számai fajonként
			\item Azok szintjei
		\end{itemize}
	\end{itemize}
	\item Játékosok kiválasztják a használandó karaktereket, majd beállítják azok felszereléseit, jelzik, ha készen vannak.
	\item Ha mindenki készen áll, akkor a szerver legenerálja a pályát. A generált pályára a játékosok elhelyezik a karaktereket. Ha ez megtörténik indul a harc.
\end{enumerate}


%----------------------------------------------------------------------------
\section{A harc menete}
%----------------------------------------------------------------------------
\begin{enumerate}
\item A szerver felállít egy sorrendet a játékosok között. A karakter fajokon végighaladva következnek a játékosok

\newcommand{\owntabnr}{4cm}
\newcommand{\colwid}{\textwidth-\owntabnr)/6}
\begin{tabular}{ |p{(\colwid}|p{(\colwid}|p{(\colwid}|p{(\colwid}|p{(\colwid}|p{(\colwid}|  }
	\hline
	\multicolumn{6}{|c|}{Példa} \\
	\hline
	\multicolumn{2}{|c|}{Lovag}& \multicolumn{2}{|c|}{Íjász} & \multicolumn{2}{|c|}{Mágus} \\
	\hline
	1. játékos lovagjai & 2. játékos lovagjai & 1. játékos lovagjai & 2. játékos lovagjai  & 1. játékos mágusa & 2. játékos mágusa \\
	\hline
\end{tabular}
\item A sorra kerülő karakter a 8 szomszédos mező egyikére lép, majd a rendszer a beállított felszerelések közül kisorsol egyet, melyet a játék felhasznál. (támadásnál a támadott ellenfelet is ki kell választani)
\item A játék addig tart, amíg már csak egy játékos karaktere(i) marad(nak) életben.
\end{enumerate}


%----------------------------------------------------------------------------
\section{A menürendszer}
%----------------------------------------------------------------------------
A menüben a játékos elindíthat egy szervert, illetve kliensként csatlakozhat egy szerverhez. Szerverként beállíthatóak a játék paraméterei a játékindítás előtt.

%----------------------------------------------------------------------------
\section{A játéktér}
%----------------------------------------------------------------------------
\begin{itemize}
	\item A harctér egy Descartes-féle derékszögű koordináta rendszer, melyben az elemek pozíciója csak egész szám lehet. ($x,y \epsilon Z  $)
	\item A harcteret a servert futtató játékos szoftvere generálja, az általa beállított \hyperref[StartParams]{\textit{paramétereknek}} megfelelően.
	\begin{itemize}
		\item Egy játékmező lehet érinthető/nem érinthető.
		\item Egy összefüggő területre (maximális N mezőre) egy dedikált játékos maximum k karaktert rakhat.
	\end{itemize}
	\item A harctéren egy mezőn, egyszerre csak egy karakter állhat.
\end{itemize}


%----------------------------------------------------------------------------
\section{Karakter}
%----------------------------------------------------------------------------
\begin{itemize}
	\item Egy karakter egy harc alatt maximum 6 felszerelést (egy fajtából akár többet is) választhat magának.
	\item A játék elején megadott karakter szint meghatározza a választható felszereléseket.
	\item A felszerelések értékei változóak, a karakterekhez tartozó összértékek a játék elején megadott paraméterekkel korlátozhatóak.
	\item A karaktereknek fajai vannak, melyek a következőek:	
	\begin{itemize}
		\item Lovag
		\item Íjász
		\item Mágus
	\end{itemize} 
\end{itemize}



%----------------------------------------------------------------------------
\section{Felszerelés}
%----------------------------------------------------------------------------
{\renewcommand{\arraystretch}{1.3}% for the vertical padding
\setstretch{0.9}

\begin{longtabu} to \textwidth {X[2l]|X[1.5l]|X[1.2c]|X[1.3c]|X[5l]}
	\rowfont[c]{\small\setstretch{0.8}}Név           & Faj     & Minimum Szint & Felhasznált érték & Leírás                                                                 \\ \hline \hline \endhead
	Fa kard       & Bármelyik & 1             & 1                 & Egy szomszédos ellenfelet támad 1 erősséggel.                          \\ \hline
	Fa pajzs      & Bármelyik & 1             & 1                 & Megvéd az 1 erősségű támadások ellen.                                  \\ \hline
	Vas kard      & Lovag     &               & 2                 & Egy szomszédos ellenfelet támad 2 erősséggel.                          \\ \hline
	Pörgő fa kard & Lovag     &               & 2                 & Az összes szomszédos ellenfelet támadja 1 erősséggel.                  \\ \hline
	Tőr           & Lovag     &               & 2                 & Ismét lép egyet, majd egy szomszédos ellenfelet támad 1 erősséggel. \\ \hline
	Vas pajzs& Lovag          &               & 2                  &  Megvéd a 2 erősségű támadások ellen.                                                                  \\ \hline
	Helytállás pengéje& Lovag          &               &  2                 & Egy szomszédos ellenfelet támad 1 erősséggel és megvéd az 1 erősségű támadások ellen.                                                                        \\ \hline
	Pörgő vas kard&  Lovag         &               &   2                &  Az összes szomszédos ellenfelet támadja 2 erősséggel.                                                                       \\ \hline
	Penge & Lovag          &               &  2                 &    Ismét lép egyet, majd egy szomszédos ellenfelet támad 2 erősséggel.                                                                     \\ \hline
	Helytállás kardja&   Lovag        &               &      3             &  Egy szomszédos ellenfelet támad 1 erősséggel és megvéd a 2 erősségű támadások ellen.  \\ \hline
	Rövid íj&Íjász &               &  2                 & Egy maximum 3 mezőnyire lévő ellenfelet támad 1 erősséggel. Nem tud lőni, ha egy szomszédos mezőn ellenfél található. \\ \hline
	Könnyű páncélzat& Íjász &               &                1   & Megvéd az 1 erősségű támadások ellen 3 körig. Pajzzsal használva a 2 erejűektől is véd.   \\ \hline
	Hosszú íj&Íjász &               &  2                 & Egy maximum 3 mezőnyire lévő ellenfelet támad 2 erősséggel. Nem tud lőni, ha egy szomszédos mezőn ellenfél található. \\ \hline
	Könnyű páncélzat& Íjász &               &                1   & Megvéd az 1 erősségű támadások ellen 5 körig. Pajzzsal használva a 2 erejűektől is véd.   \\ \hline
	Csizma& Íjász &               &                  3 & Ismét lép egyet, majd újrasorsol egy felszerelést  \\ \hline
	Fagyasztó kristályok& Mágus&               &    1               & Egy maximum 2 mezőnyire lévő ellenfelet kiválasztva, az kimarad 2 körből. \\ \hline
	Tűzgolyó& Mágus &               &  2                 &Egy maximum 4 mezőnyire lévő ellenfelet támad 1 erősséggel.  \\ \hline
	Villámcsapás& Mágus &               &  1                 &  Egy maximum 3 mezőnyire lévő ellenfelet és annak egy szomszédját támadja 1 erősséggel. \\\hline
	Tűzvihar& Mágus &               &  3                 & Egy maximum 6 mező távolságra lévő mezőt és annak környezetét támadja 1 erősséggel. \\ \hline
	Meteor& Mágus &               &  3                 & Egy maximum 6 mező távolságra lévő mezőt 2, és annak környezetét támadja 1 erősséggel.  \\ \hline 
\end{longtabu}