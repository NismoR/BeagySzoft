%----------------------------------------------------------------------------
%\chapter{\LaTeX-eszközök}\label{sect:LatexTools}
%----------------------------------------------------------------------------
\section{Általános leírás}
%----------------------------------------------------------------------------
A megvalósítani kívánt játék egy körökre osztott stratégiai játék, mely során egy harctéren különböző fajú karakterekkel harcolnak a játékosok.
A karakterek a játék előtt meghatározott felszerelést visznek magukkal, melyeket felhasználhatják a harc során. Körönként egy karakter egyet lép, majd egy kisorsolt felszerelést felhasznál. A játék addig tart, amíg már csak egy játékos karaktere(i) marad(nak) életben.


%----------------------------------------------------------------------------
\section{A játék menete}
%----------------------------------------------------------------------------
\begin{enumerate}
	\item Egyik játékos elindítja a Servert, a másik (többi) kliensként csatlakozik.
	\item Server játékos beállítja a játék paramétereit.
	\begin{itemize}\label{StartParams}
		\item Maximális játékos szám, maximális karakter száma játékosonként.
		\item Maximális pályaméret(szélesség és magasság).
		\item Használható karakter fajok
		\begin{itemize}
			\item Azok számai fajonként
			\item Azok szintjei
		\end{itemize}
	\end{itemize}
	\item Játékosok kiválasztják a használandó karaktereket, majd beállítják azok felszereléseit, jelzik, ha készen vannak.
	\item Ha mindenki készen áll, akkor a szerver legenerálja a pályát. A generált pályára a játékosok elhelyezik a karaktereket. Ha ez megtörténik indul a harc.
\end{enumerate}


%----------------------------------------------------------------------------
\section{A harc menete}
%----------------------------------------------------------------------------
\begin{enumerate}
\item A szerver felállít egy sorrendet a játékosok között. A karakter fajokon végighaladva következnek a játékosok

\newcommand{\owntabnr}{4cm}
\newcommand{\colwid}{\textwidth-\owntabnr)/6}
\begin{tabular}{ |p{(\colwid}|p{(\colwid}|p{(\colwid}|p{(\colwid}|p{(\colwid}|p{(\colwid}|  }
	\hline
	\multicolumn{6}{|c|}{Példa} \\
	\hline
	\multicolumn{2}{|c|}{Lovag}& \multicolumn{2}{|c|}{Íjász} & \multicolumn{2}{|c|}{Mágus} \\
	\hline
	1. játékos lovagjai & 2. játékos lovagjai & 1. játékos lovagjai & 2. játékos lovagjai  & 1. játékos mágusa & 2. játékos mágusa \\
	\hline
\end{tabular}
\item A sorra kerülő karakter a 8 szomszédos mező egyikére lép, majd a rendszer a beállított felszerelések közül kisorsol egyet, melyet a játék felhasznál. (támadásnál a támadott ellenfelet is ki kell választani)
\item A játék addig tart, amíg már csak egy játékos karaktere(i) marad(nak) életben.
\end{enumerate}


%----------------------------------------------------------------------------
\section{A menürendszer}
%----------------------------------------------------------------------------


%----------------------------------------------------------------------------
\section{A játéktér}
%----------------------------------------------------------------------------
\begin{itemize}
	\item A harctér egy Descartes-féle derékszögű koordináta rendszer, melyben az elemek pozíciója csak egész szám lehet. ($x,y \epsilon Z  $)
	\item A harcteret a servert futtató játékos szoftvere generálja, az általa beállított \hyperref[StartParams]{\textit{paramétereknek}} megfelelően.
	\begin{itemize}
		\item Egy játékmező lehet érinthető/nem érinthető.
		\item Egy összefüggő területre (maximális N mezőre) egy dedikált játékos maximum k karaktert rakhat.
	\end{itemize}
	\item A harctéren egy mezőn, egyszerre csak egy karakter állhat.
\end{itemize}


%----------------------------------------------------------------------------
\section{Karakter}
%----------------------------------------------------------------------------
\begin{itemize}
	\item Egy karakter egy harc alatt maximum 6 felszerelést (egy fajtából akár többet is) választhat magának.
	\item A játék elején megadott karakter szint meghatározza a választható felszereléseket.
	\item A felszerelések értékei változóak, a karakterekhez tartozó összértékek a játék elején megadott paraméterekkel korlátozhatóak.
	\item A karaktereknek fajai vannak, melyek a következőek:	
	\begin{itemize}
		\item Lovag
		\item Íjász
		\item Mágus
	\end{itemize} 
\end{itemize}



%----------------------------------------------------------------------------
\section{Felszerelés}
%----------------------------------------------------------------------------
% Please add the following required packages to your document preamble:
% \usepackage{graphicx}
\begin{center}
%\begin{tabular}[]
%	\centering
%	\label{my-label}
%	\resizebox{\textwidth}{!}{%
		\begin{tabular}{lllll}
			Név           & Faj     & Minimum Szint & Felhasznált érték & Leírás                                                                 \\ \hline
			Fa kard       & Bármelyik & 1             & 1                 & Egy szomszédos ellenfelet támad 1 erősséggel.                          \\
			Fa pajzs      & Bármelyik & 1             & 1                 & Megvéd az 1 erősségű támadások ellen.                                  \\
			Vas kard      & Lovag     &               & 2                 & Egy szomszédos ellenfelet támad 2 erősséggel.                          \\
			Pörgő fa kard & Lovag     &               & 2                 & Az összes szomszédos ellenfelet támadja 1 erősséggel.                  \\
			Tőr           & Lovag     &               & 2                 & Ismét léphet egyet, majd egy szomszédos ellenfelet támad 1 erősséggel. \\
			&           &               &                   &                                                                        \\
			&           &               &                   &                                                                        \\
			&           &               &                   &                                                                        \\
			&           &               &                   &                                                                        \\
			&           &               &                   &                                                                       
		\end{tabular}%
%	}
%\end{tabular}
\end{center}